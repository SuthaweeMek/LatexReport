%%%%%% Run at command line, run
%%%%%% xelatex grad-sample.tex 
%%%%%% for a few times to generate the output pdf file
\documentclass[12pt,oneside,openright,a4paper]{cpe-thai-project}


\defaultfontfeatures{Mapping=tex-text,Scale=1.23,LetterSpace=0.0}
\setmainfont[Scale=1.23,LetterSpace=0,WordSpace=1.0,FakeStretch=1.0]{TH Sarabun New}
%\setmathfont(Digits)[Scale=1.0,LetterSpace=0,FakeStretch=1.0]{Times New Roman}
\setlength{\parindent}{4em} 
%%%%%%%%%%%%%%%%%%%%%%%%%%%%%%%%%%%%%%%%%%%%%%%%%%%%%%%%%%%%%%%%%%%
% Customize below to suit your needs 
% The ones that are optional can be left blank. 
%%%%%%%%%%%%%%%%%%%%%%%%%%%%%%%%%%%%%%%%%%%%%%%%%%%%%%%%%%%%%%%%%%%
% First line of title
\def\disstitleone{Project/Indep study title line 1}   
% Second line of title
\def\disstitletwo{Project/Indep title line 2 (optional)}   
% Your first name and lastname
\def\dissauthor{Mr./Ms. Firstname1 Lastname1}   % 1st member
%%% Put other group member names here ..
\def\dissauthortwo{Mr./Ms. Firstname2 Lastname2}   % 2nd member (optional)
\def\dissauthorthree{Mr./Ms. Firstname3 Lastname3}   % 3rd member (optional)


% The degree that you're persuing..
\def\dissdegree{Bachelor of Engineering} % Name of the degree
\def\dissdegreeabrev{B.Eng} % Abbreviation of the degree
\def\dissyear{202x}                   % Year of submission
\def\thaidissyear{256x}               % Year of submission (B.E.)

%%%%%%%%%%%%%%%%%%%%%%%%%%%%%%%%%%%%%%%%%%%%
% Your project and independent study committee..
%%%%%%%%%%%%%%%%%%%%%%%%%%%%%%%%%%%%%%%%%%%%
\def\dissadvisor{Assoc.Prof. My main advisor name , Ph.D.}  % Advisor
%%% Leave it empty if you have no Co-advisor
\def\disscoadvisor{Assoc.Prof. My Co-advisor name, Ph.D.}  % Co-advisor
\def\disscommitteetwo{Asst.Prof. Committee2, Ph.D.}  % 3rd committee member (optional)
\def\disscommitteethree{Asst.Prof. Committee3, Ph.D.}   % 4th committee member (optional) 
\def\disscommitteefour{}    % 5th committee member (optional) 

\def\worktype{Project} %%  Project or Independent study
\def\disscredit{3}   %% 3 credits or 6 credits


\def\fieldofstudy{Computer Engineering} 
\def\department{Computer Engineering} 
\def\faculty{Engineering}

\def\thaifieldofstudy{วิศวกรรมคอมพิวเตอร์} 
\def\thaidepartment{วิศวกรรมคอมพิวเตอร์} 
\def\thaifaculty{วิศวกรรมศาสตร์}
 
\def\appendixnames{Appendix} %%% Appendices or Appendix

\def\thaiworktype{ปริญญานิพนธ์} %  Project or research project % 
\def\thaidisstitleone{หัวข้อปริญญานิพนธ์บรรทัดแรก}
\def\thaidisstitletwo{หัวข้อปริญญานิพนธ์บรรทัดสอง}
\def\thaidissauthor{นายสมศักดิ์ คอมพิวเตอร์}
\def\thaidissauthortwo{นางสาวสมศรี คอมพิวเตอร์2} %Optional
\def\thaidissauthorthree{นางสาวสมปอง คอมพิวเตอร์3} %Optional

\def\thaidissadvisor{รศ.ดร.ที่ปรึกษา วิทยานิพนธ์}
%% Leave this empty if you have no co-advisor
\def\thaidisscoadvisor{รศ.ดร.ที่ปรึกษา วิทยานิพนธ์ร่วม} %Optional
\def\thaidissdegree{วิศวกรรมศาสตรบัณฑิต}

% Change the line spacing here...
\linespread{1.15}

%%%%%%%%%%%%%%%%%%%%%%%%%%%%%%%%%%%%%%%%%%%%%%%%%%%%%%%%%%%%%%%%
% End of personal customization.  Do not modify from this part 
% to \begin{document} unless you know what you are doing...
%%%%%%%%%%%%%%%%%%%%%%%%%%%%%%%%%%%%%%%%%%%%%%%%%%%%%%%%%%%%%%%%


%%%%%%%%%%%% Dissertation style %%%%%%%%%%%
%\linespread{1.6} % Double-spaced  
%%\oddsidemargin    0.5in
%%\evensidemargin   0.5in
%%%%%%%%%%%%%%%%%%%%%%%%%%%%%%%%%%%%%%%%%%%
%\renewcommand{\subfigtopskip}{10pt}
%\renewcommand{\subfigbottomskip}{-5pt} 
%\renewcommand{\subfigcapskip}{-6pt} %vertical space between caption
%                                    %and figure.
%\renewcommand{\subfigcapmargin}{0pt}

\renewcommand{\topfraction}{0.85}
\renewcommand{\textfraction}{0.1}

\newtheorem{theorem}{Theorem}
\newtheorem{lemma}{Lemma}
\newtheorem{corollary}{Corollary}

\def\QED{\mbox{\rule[0pt]{1.5ex}{1.5ex}}}
\def\proof{\noindent\hspace{2em}{\itshape Proof: }}
\def\endproof{\hspace*{\fill}~\QED\par\endtrivlist\unskip}
%\newenvironment{proof}{{\sc Proof:}}{~\hfill \blacksquare}
%% The hyperref package redefines the \appendix. This one 
%% is from the dissertation.cls
%\def\appendix#1{\iffirstappendix \appendixcover \firstappendixfalse \fi \chapter{#1}}
%\renewcommand{\arraystretch}{0.8}
%%%%%%%%%%%%%%%%%%%%%%%%%%%%%%%%%%%%%%%%%%%%%%%%%%%%%%%%%%%%%%%%
%%%%%%%%%%%%%%%%%%%%%%%%%%%%%%%%%%%%%%%%%%%%%%%%%%%%%%%%%%%%%%%%
\begin{document}

\makesignaturepage 

%%%%%%%%%%%%%%%%%%%%%%%%%%%%%%%%%%%%%%%%%%%%%%%%%%%%%%%%%%%%%%
%%%%%%%%%%%%%%%%%%%%%% English abstract %%%%%%%%%%%%%%%%%%%%%%%
%%%%%%%%%%%%%%%%%%%%%%%%%%%%%%%%%%%%%%%%%%%%%%%%%%%%%%%%%%%%%%
\abstract

In a multihop ad hoc network, the interference among nodes is
  reduced to maximize the throughput by using a smallest transmission
  range that still preserve the network connectivity. However, most
  existing works on transmission range control focus on the
  connectivity but lack of results on the throughput performance. This
  paper analyzes the per-node saturated throughput of an IEEE 802.11b
  multihop ad hoc network with a uniform transmission range. Compared
  to simulation, our model can accurately predict the per-node
  throughput.  The results show that the maximum achievable per-node
  throughput can be as low as 11\% of the channel capacity in a normal
  set of $\alpha$ operating parameters independent of node density. However, if
  the network connectivity is considered, the obtainable throughput
  will reduce by as many as 43\% of the maximum throughput. 

\begin{flushleft}
\begin{tabular*}{\textwidth}{@{}lp{0.8\textwidth}}
\textbf{Keywords}: & Multihop ad hoc networks / Topology control / Single-Hop Throughput
\end{tabular*}
\end{flushleft}
\endabstract

%%%%%%%%%%%%%%%%%%%%%%%%%%%%%%%%%%%%%%%%%%%%%%%%%%%%%%%%%%%%%%
%%%%%%%%%% Thai abstract here %%%%%%%%%%%%%%%%%%%%%%%%%%%%%%%%%
%%%%%%%%%%%%%%%%%%%%%%%%%%%%%%%%%%%%%%%%%%%%%%%%%%%%%%%%%%%%%%
{\newfontfamily\thaifont{TH Sarabun New:script=thai}[Scale=1.3]
\XeTeXlinebreaklocale "th_TH"	
\thaifont
\thaiabstract

การวิจัยครั้งนี้มีวัตถุประสงค์  เพื่อศึกษาความพึงพอใจในการให้บริการงานทั่วไปของสานักวิชา พื้นฐานและภาษา เพื่อเปรียบเทียบระดับความพึงพอใจต่อการให้บริการงาน ทั่วไปของสานักวิชาพื้นฐานและภาษา ของนักศึกษาที่มาใช้บริการสานักวิชาพื้นฐานและภาษา สถาบัน เทคโนโลยีไทย-ญี่ปุ่น จาแนกตามเพศ คณะ และชั้นปีที่ศึกษา เพื่อศึกษาปัญหาและข้อเสนอแนะของ นักศึกษามาเป็นแนวทางในการพัฒนาและปรับปรุงการให้บริการของสานักวิชาพื้นฐานและภาษา

\begin{flushleft}
\begin{tabular*}{\textwidth}{@{}lp{0.8\textwidth}}
 & \\

\textbf{คำสำคัญ}: & การชุบเคลือบด้วยไฟฟ้า / การชุบเคลือบผิวเหล็ก /  เคลือบผิวรังสี
\end{tabular*}
\end{flushleft}
\endabstract
}

%%%%%%%%%%%%%%%%%%%%%%%%%%%%%%%%%%%%%%%%%%%%%%%%%%%%%%%%%%%%
%%%%%%%%%%%%%%%%%%%%%%% Acknowledgments %%%%%%%%%%%%%%%%%%%%
%%%%%%%%%%%%%%%%%%%%%%%%%%%%%%%%%%%%%%%%%%%%%%%%%%%%%%%%%%%%
\preface
ขอบคุณอาจารย์ที่ปรึกษา กรรมการ พ่อแม่พี่น้อง และเพื่อนๆ คนที่ช่วยให้งานสำเร็จ ตามต้องการ

%%%%%%%%%%%%%%%%%%%%%%%%%%%%%%%%%%%%%%%%%%%%%%%%%%%%%%%%%%%%%
%%%%%%%%%%%%%%%% ToC, List of figures/tables %%%%%%%%%%%%%%%%
%%%%%%%%%%%%%%%%%%%%%%%%%%%%%%%%%%%%%%%%%%%%%%%%%%%%%%%%%%%%%
% The three commands below automatically generate the table 
% of content, list of tables and list of figures
\tableofcontents                    
\listoftables
\listoffigures                      

%%%%%%%%%%%%%%%%%%%%%%%%%%%%%%%%%%%%%%%%%%%%%%%%%%%%%%%%%%%%%%
%%%%%%%%%%%%%%%%%%%%% List of symbols page %%%%%%%%%%%%%%%%%%%
%%%%%%%%%%%%%%%%%%%%%%%%%%%%%%%%%%%%%%%%%%%%%%%%%%%%%%%%%%%%%%
% You have to add this manually..
%\listofsymbols
%\begin{flushleft}
%\begin{tabular}{@{}p{0.07\textwidth}p{0.7\textwidth}p{0.1\textwidth}}
%\textbf{SYMBOL}  & & \textbf{UNIT} \\[0.2cm]
%$\alpha$ & Test variable\hfill & m$^2$ \\
%$\lambda$ & Interarival rate\hfill &  jobs/second\\
%$\mu$ & Service rate\hfill & jobs/second\\
%\end{tabular}
%\end{flushleft}
%%%%%%%%%%%%%%%%%%%%%%%%%%%%%%%%%%%%%%%%%%%%%%%%%%%%%%%%%%%%%%
%%%%%%%%%%%%%%%%%%%%% List of vocabs & terms %%%%%%%%%%%%%%%%%
%%%%%%%%%%%%%%%%%%%%%%%%%%%%%%%%%%%%%%%%%%%%%%%%%%%%%%%%%%%%%%
% You also have to add this manually..
%\listofvocab
%\begin{flushleft}
%\begin{tabular}{@{}p{1in}@{=\extracolsep{0.5in}}l}
%ABC & Adaptive Bandwidth Control \\
%MANET & Mobile Ad Hoc Network 
%\end{tabular}
%\end{flushleft}

%\setlength{\parskip}{1.2mm}

%%%%%%%%%%%%%%%%%%%%%%%%%%%%%%%%%%%%%%%%%%%%%%%%%%%%%%%%%%%%%%%
%%%%%%%%%%%%%%%%%%%%%%%% Main body %%%%%%%%%%%%%%%%%%%%%%%%%%%%
%%%%%%%%%%%%%%%%%%%%%%%%%%%%%%%%%%%%%%%%%%%%%%%%%%%%%%%%%%%%%%%


\chapter{บทนำ}






\section{ที่มาและความสำคัญ}

\par โรคการบกพร่องทางการเรียนรู้ในเด็ก (Learning disorder, LD) คือ ความผิดปกติทางการเรียนรู้ที่เกิดจากการทำงานผิดปกติของสมอง ทำให้ผลการเรียนของเด็กต่ำกว่าศักยภาพที่แท้จริง โดยแบ่งออกเป็น 3 ประเภทตามความผิดปกติของกระบวนการเรียนรู้ที่แสดงออก นั่นคือ ความบกพร่องด้านการอ่าน ความบกพร่องทางด้านการเขียนสะกดคำ และความบกพร่องทางด้านคณิตศาสตร์ โดยเด็กที่มีความบกพร่องด้านการอ่านจะไม่สามารถจดจำพยัญชนะ สระ และยังไม่สามารถสะกดคำได้จึงเป็นสาเหตุให้  เกิดการอ่านออกเสียงไม่ชัด ไม่สามารถผันวรรณยุกตร์ได้ หรืออ่านไม่ออก ส่วนความบกพร่องด้านที่สอง คือ   การเขียนสะกดคำ ความบกพร่องด้านนี้สามารถพบได้ร่วมกับความบกพร่องด้านการอ่าน เด็กมีความบกพร่องในการสะกดพยัญชนะ สระ หรือ วรรณยุกต์ จึงทำให้เกิดการเขียนหนังสือที่ไม่ถูกต้อง และความบกพร่องสุดท้ายคือ ความบกพร่องด้านคณิตศาสตร์ ลักษณะของเด็กประเภทนี้คือ ขาดทักษะการเข้าใจตัวเลข และจะเกิดการนับจำนวนหรือบวกคูณลบเลขผิด จึงไม่สามารถทำให้คำนวณเลขได้ สาเหตุของโรคการบกพร่องทางการเรียนรู้ที่เกิดจากการทำงานผิดปกติของสมองมีได้หลายสาเหตุด้วยกัน เช่น การทำงานของสมองบางตำแหน่งบกพร่อง กรรมพันธุ์ หรือความผิดปกติของโครโมโซม อ้างอิงจากข้อมูลที่ได้มาจาก พญ.วินัดดา ปิยะศิลป์ในพ.ศ. 2554 คาดว่ามีเด็กที่เป็นโรคการบกพร่องทางการเรียนรู้ หรือ LD (Learning  Disorders)  ประมาณ 500,000 คน ในช่วงที่เก็บข้อมูลสถิตินั้นมีอัตราเด็กเกิดใหม่ถึง 800,000 คนต่อปี แล้วคาดว่ามีโอกาสที่เด็กเป็น LD 40,000 คนต่อปี จากข้อมูลข้างต้นทำให้ทราบว่าเด็กที่เป็นโรคการบกพร่องทางการเรียนรู้มีจำนวนมาก  โดยในปัจจุบันเด็กสามารถเข้ารับการทำแบบทดสอบเพื่อวินิจฉัยโรคบกพร่องทางการเรียนรู้ได้ ซึ่งจะมีบุคลากรทางการแพทย์ควบคุมการทำแบบทดสอบและจำเป็นต้องให้แพทย์ผู้เชี่ยวชาญเป็นผู้วินิจฉัย กระบวนการนี้ใช้ระยะเวลานาน เนื่องจากบุคลากรการแพทย์มีจำกัด ทำให้ไม่สามารถรองรับเด็กเข้ามาทำแบบทดสอบได้เป็นจำนวนมากต่อวัน ซึ่งหากเด็กได้รับการรักษาที่ล่าช้า อาจจะทำให้ได้ผลลัพธ์การรักษาน้อยลง
\par จากสาเหตุข้างต้นจึงทำให้กลุ่มผู้พัฒนาจึงนำเสนอ “แอลดีสปอต หรือ ระบบตรวจจับอาการโรคการบกพร่องทางการเรียนรู้ทางด้านการเขียนสะกดคำ” ผ่านทางภาพการเขียนตัวอักษร สระ และ สะกดคำโดยใช้แอปพลิเคชันซึ่งเด็กต้องทำแบบทดสอบในรูปแบบของเกมด้วยการเขียนตัวอักษร สระ และสะกดคำ จากนั้นภาพแบบทดสอบจะถูกส่งให้ระบบแอลดีสปอต เพื่อคำนวณคะแนนและวินิจฉัยโรคการบกพร่องทางการเรียนรู้เบื้องต้น และนำไปแสดงผลในแอปพลิเคชันให้บุคลากรทางการแพทย์และผู้ปกครองสามารถดูผลลัพธ์ได้ ซึ่งในส่วนของการวินิจฉัยนั้นได้อ้างอิงหลักการวิเคราะห์ข้อมูลจากแพทย์มาใช้ในระบบวิเคราะห์ที่จะพัฒนา
\par แอลดีสปอต นั้นจะช่วยให้การวินิจฉัยโรคการบกพร่องทางการเรียนรู้เบื้องต้นในเด็กสามารถเข้าถึงได้ง่ายขึ้นโดยที่เด็กจะสามารถทำแบบทดสอบเบื้องต้นได้ผ่านทางแอปพลิเคชันก่อนที่จะเดินทางมาที่โรงพยาบาลเพื่อที่จะลดความซับซ้อนและระยะเวลาในการรอการวินิจฉัยเบื้องต้น อีกทั้งยังลดขั้นตอนหรือหน้าที่ของแพทย์หรือบุคลากร





\section{วัตถุประสงค์}

\begin{itemize}
  \item  เพื่อพัฒนาระบบวิเคราะห์รูปภาพลายมือเขียนของเด็กเพื่อวินิจฉัยโอกาสเป็นโรคการบกพร่องทางการเรียนรู้ด้านการเขียนและสะกดคำในเด็กได้อย่างแม่นยำ
  \item  เพื่อพัฒนาแอปพลิเคชันที่อยู่ในรูปแบบเกมส์เพื่อดึงดูดความสนใจจากเด็ก และเด็กสามารถทำแบบทดสอบจนจบได้ 
  \item  เพื่อลดความซับซ้อนและระยะเวลาในการรอเพื่อวินิจฉัยโรคการบกพร่องทางการเรียนรู้เบื้องต้นได้
  \item  เพื่อช่วยให้บุคลากรทางการแพทย์สามารถทำงานได้อย่างมีประสิทธิภาพมากยิ่งขึ้น  
  \end{itemize}

\section{ขอบเขตของโครงงาน}

\begin{itemize}
\item  แอปพลิเคชั่นในรูปแบบของเกมส์ที่รองรับระบบปฏิบัติการแอนดรอยด์ (Android)  และ ไอโอเอส (IOS) ซึ่งรองรับเพียงภาษาไทย 
\item  ระบบวิเคราะห์รูปภาพลายมือของเด็กซึ่งถูกสร้างขึ้นมาจากข้อมูลแบบทดสอบการเขียนของเด็กที่เป็นโรคการบกพร่องทางการเรียนรู้ จากหน่วยตรวจโรคจิตเวชเด็กและวัยรุ่น ภาควิชาจิตเวชศาสตร์ คณะแพทยศาสตร์ศิริราชพยาบาล
\item  ระบบวิเคราะห์รูปภาพลายมือเขียนของเด็กจะต้องรับรูปภาพลายมือของเด็ก โดยการเขียนผ่านทางแอปพลิเคชั่นที่ได้สร้างไว้
\item  ผลลัพธ์จะออกมาในรูปแบบจำนวนความผิดพลาดจากที่เขียนผิด และความน่าจะเป็นว่าเด็กมีความน่าจะเป็นที่โรคการบกพร่องทางการเรียนรู้เท่าใด โดยตัวระบบจะเรียนรู้จากภาพการเขียนทดสอบของเด็กที่เป็นโรคการบกพร่องทางการเรียนรู้และภาพการเขียนทดสอบของเด็กที่ไม่เป็นโรคการบกพร่องทางการเรียนรู้ จาก หน่วยตรวจโรคจิตเวชเด็กและวัยรุ่น ภาควิชาจิตเวชศาสตร์ คณะแพทยศาสตร์ศิริราชพยาบาล
\item  ระบบจะแสดงผลลัพธ์ที่ได้จากการวินิจฉัยในแอปพลิเคชัน โดยที่ผู้ปกครองและบุคลากรทางกาารแพทย์จะสามารถเข้ามาดูผลลัพธ์แล้วนำไปใช้ประโยชน์ต่อได้ 
\end{itemize}

\section{ประโยชน์ที่คาดว่าจะได้่รับ}

โครงงานนี้จะเป็นประโยชน์กับใคร ยังไง ทั้งในเชิงรูปธรรมและนามธรรม ในปัจจุบันหรือในอนาคตถ้านำไป
ต่อยอด

\section{ตารางการดำเนินงาน}

%%%%%%%%%%%%%%%%%%%%%%%%%%%%%%%%%%%%%%%%%%%%%%%%%%%%%%%%%%%%
%%%%%%%%%%%%%%  Literature Review %%%%%%%%%%%%%%%%%%%%%%%%%%
%%%%%%%%%%%%%%%%%%%%%%%%%%%%%%%%%%%%%%%%%%%%%%%%%%%%%%%%%%%%
\chapter{ทฤษฎีความรู้และงานที่เกี่ยวข้อง}

อธิบายทฤษฎี องค์ความรู้หลักที่ใช้ในงาน งานวิจัยที่นำมาใช้ในโครงงาน หรือเปรียบเทียบผลิตภัณฑ์ที่มีอยู่ในท้องตลาด
Explain theory, algorithms, protocols, or existing research works and tools related to your work. 

\section{Core concept แนวคิดหลัก}
เนื่องจากตัวระบบที่เราสร้างขึ้นมาเพื่อวินิจฉัยโรคการเรียนรู้บกพร่องในเด็กนั้นจะต้องทำการเรียนรู้ข้อมูลลักษณะจุดเด่นต่างๆของภาพผลแบบทดสอบการเรียนรู้บกพร่องในเด็กว่า 
มีลักษณะเด่นใดจึงจำแนกว่าเด็กคนนั้นมีโอกาสเป็นโรคการเรียนรู้บกพร่องในเด็ก 
จากการค้นคว้าหาข้อมูลจึงพบว่าเราจำเป็นที่จะต้องใช้ความรู้ในเรื่องของ Convolutional Neural Network 
ซึ่งเหมาะแก่การทำการจำแนกประเภทของรูปภาพ และเป็นส่วนหนึ่งของเรื่องการเรียนรู้เชิงลึกของคอมพิวเตอร์ (deep learning)

\subsection{การเรียนรู้เชิงลึกของคอมพิวเตอร์ (deep learning)}

\par การเรียนรู้เชิงลึกของคอมพิวเตอร์ (deep learning) เป็นหนึ่งในสาขาย่อยของ machine learning เป็นศาสตร์ที่พูดถึงการจำลองการทำงานของระบบโครงข่ายประสาทมนุษย์ โดย จะมีการแบ่งการทำงานข้างในเป็น layer ต่างๆ โดยเราจะมองเป็นสามส่วนหลักๆได้แก่ 	

\begin{enumerate}
  \item Input Layer มีหน้าที่สำหรับการรับข้อมูลป้อนเข้าโครงข่ายประสาท จากผู้ใช้ เช่น รูปภาพ
  \item Hidden Layer มีหน้าที่สำหรับการประเมินข้อมูลที่ป้อนเข้ามา เพื่อหาข้อมูลต่างๆที่ใช้ในการจำแนกประเภทโดยตัว Hidden Layer นั้นสามารถมีได้มากกว่า 1 ชั้น
  \item Output layer เป็นชั้นสุดท้ายมีหน้าที่สำหรับรับข้อมูลจาก Hidden Layer เพื่อใช้ในการบอกว่าท้ายที่สุดแล้วตัวข้อมูลที่รับเข้ามานั้นถูกจำแนกอยู่ในประเภทใด
\end{enumerate}

\begin{figure}[!ht]\centering
  \setlength{\fboxrule}{0.2mm} % can define this in the preamble
  \setlength{\fboxsep}{1cm}
  \fbox{\includegraphics[width=10cm]{./DeepLearning.jpg}}
  \caption{ตัวอย่าง layer ของ การเรียนรู้เชิงลึกของคอมพิวเตอร์}\label{fig:deep}
  \source{[ที่มา : https://verneglobal.com/news/blog/deep-learning-at-scale]}
\end{figure}

\subsection{โครงข่ายประสาทเทียมแบบสังวัตนาการ (Convolutional Neural Network)}
\FloatBarrierในปัจจุบันการทำ การจำแนกประเภทรูปภาพ สำหรับทางด้านการแพทย์กำลังเป็นที่สนใจ โครงข่ายประสาทเทียมแบบสังวัตนาการ 
หรือ CNN เลยได้รับความนิยมมากขึ้นโดย CNN เป็นรูปแบบหนึ่งของ การเรียนรู้เชิงลึกของคอมพิวเตอร์ ที่เกิดจากการนำแนวคิดของ  
Neural Network มาเพิ่มในส่วนของ Convolutional layer ซึ่งเหมาะแก่การหาลักษณะต่างๆของข้อมูลต่างๆ เช่น รูปภาพ 
โดยตัวของ CNN นั้นจะประกอบด้วยหลายๆ layer ด้วยกัน

\begin{figure}[!ht]\centering
  \setlength{\fboxrule}{0.2mm} % can define this in the preamble
  \setlength{\fboxsep}{1cm}
  \fbox{\includegraphics[width=10cm]{./cnn.jpg}}
  \caption{แสดงตัวอย่างโครงข่าย CNN ที่ประกอบด้วย 1. convolutional layer 2. pooling layer 3. fully connected layer
  }\label{fig:cnn}
  \source{[ที่มา : https://developers.google.com/machine-learning/practica/image-classification/convolutional-neural-networks]}
\end{figure}

โดย CNN จะมี layer หลักๆได้แก่ 	
\begin{enumerate}
  \item Convolutional layer ซึ่งมีหน้าที่ในการประมวลผลภาพเพื่อหาคุณลักษณะต่างๆ เช่น สี ขอบ ด้วย filters และนำไปเข้า activate function เพื่อแปลงผลลัพธ์ให้กลายเป็นข้อมูลนำเข้าสำหรับ layer ถัดไป 
  โดยมี activate function ที่ได้รับความนิยมในการทำ การจำแนกประเภทรูปภาพ คือ Rectified Linear Unit (ReLU)
  \item Pooling layer มีหน้าที่ในการลดมิติของข้อมูลที่เราได้จาก Convolutional layer ให้เล็กลง และคงไว้ซึ่งข้อมูลที่จำเป็นเพื่อที่จะทำให้การประมวลผลเร็วขึ้น โดยจะมีสองวิธีหลักๆได้แก่ 
  Max pooling และ Mean pooling โดย Max pooling จะทำการเลือกค่าที่มากที่สุดในขอบเขตที่สนใจ และ Mean Pooling จะทำการหาค่าเฉลี่ยของขอบเขตที่สนใจแล้วนำไปใช้ต่อ ดังรูป \ref{fig:pooling} 
  \begin{figure}[!ht]\centering
    \setlength{\fboxrule}{0.2mm} % can define this in the preamble
    \setlength{\fboxsep}{1cm}
    \fbox{\includegraphics[width=10cm]{./pooling.png}}
    \caption{ตัวอย่างการทำ Max pooling และ mean pooling}\label{fig:pooling}
    \source{[ที่มา : https://stackoverflow.com/questions/44287965]}
  \end{figure}
  \item fully connected layer มีหน้าที่ในการรวบรวม output จาก layer ก่อนหน้า ที่ได้ทำการหา คุณลักษณะต่างๆมารวมและกำหนดให้ผลลัพธ์ของ layer นี้มีจำนวนเท่ากับ จำนวนประเภทที่เราต้องการจำแนกรูปภาพ 
  เพื่อดูว่าผลลัพธ์ท้ายสุดเราจำแนกรูปภาพนั้นได้อยู่ในประเภทไหนซึ่ง CNN จะประกอบด้วยหลายๆ layer นี้เรียงกันไปมาจนถึง output ตามความเหมาะสมของ โมเดลนั้นๆ และสามารถปรับ parameter ของแต่ละ layer ได้เพื่อทำให้การจำแนกประเภทนั้นออกมาแม่นยำที่สุด 
  โดยในโครงการนี้ เราจะเลือกใช้ Convolutional neural network ในการสร้าง โมเดลเพื่อจำแนกข้อมูลภาพถ่ายของเรา
\end{enumerate}

\subsection{Transfer Learning}
ในการทำ Convolutional neural networkนั้น เราจำเป็นจะต้องออกแบบตัว layer และ parameter ต่างๆให้เหมาะสมเพื่อให้ได้ความแม่นยำในการจำแนกที่สูง
ซึ่งเราจำเป็นต้องใช้ข้อมูลจำนวนมากในการ train ให้โมเดล CNN ของเรานั้นมีความแม่นยำ แต่ว่า Transfer Learning คือการที่เรานำ โมเดล CNN ที่มีการสร้างขึ้นมาไว้แล้วจากข้อมูลอื่น มาปรับแต่งในส่วนของ
fully connected layer เองใหม่ให้เหมาะสมกับข้อมูลที่เราจะทำการจำแนก ซึ่งจะทำให้เราประหยัดเวลาในการสร้างโมเดล และลดจำนวนข้อมูลที่ใช้ในการสร้างโมเดลเพื่อที่จะทำให้โมเดลนั้นมีความแม่นยำ
\begin{figure}[!ht]\centering
  \setlength{\fboxrule}{0.2mm} % can define this in the preamble
  \setlength{\fboxsep}{1cm}
  \fbox{\includegraphics[width=10cm]{./transfer.jpg}}
  \caption{ภาพอธิบายตัวอย่างของ Transfer Learning}\label{fig:transfer}
  \source{[ที่มา : https://www.topbots.com/transfer-learning-in-nlp/]}
\end{figure}

\newpage 
\subsection{Activate Function}
Activate function มีหน้าที่ในการปรับผลลัพธ์ ของ neuron ในแต่ละ layer ก่อนจะส่งต่อไปเป็นข้อมูลนำเข้าสำหรับ layer ถัดไป โดย Activate function ที่เป็นที่นิยมคือ sigmoid function เนื่องจากตัว sigmoid function 
น้ันจะมีผลลัพธ์ออกมาอยู่ในช่วงของ 0 จนถึง 1 ทำให้เหมาะแก่การใช้ทำเรื่องความน่าจะเป็น แต่เนื่องจากกราฟของ sigmoid function เป็นดังรูป \ref{fig:sigmoid}  เราจะเห็นว่าหากค่า |x| มีค่าสูงมากขึ้นค่าของ sigmoid tfunction จะมีการเปลี่ยนแปลงที่น้อยลง 
หรือมีค่าอนุพันธ์ที่น้อยลงทำให้การอัพเดทน้ำหนักของตัว Neural network ใน layer แ รกๆนั้นมีค่าน้อยจนอาจทำให้การเรียนรู้หยุด  
ปัญหานี้มีชื่อเรียกว่า Vanishing gradient problem โดยสามารถแก้ไขด้วยการเปลี่ยน activate function ได้ยกตัวอย่างเช่น Rectified Linear Unit หรือ ReLU

\begin{figure}[!ht]\centering
  \setlength{\fboxrule}{0.2mm} % can define this in the preamble
  \setlength{\fboxsep}{1cm}
  \fbox{\includegraphics[width=10cm]{./sigmoid.jpg}}
  \caption{ภาพตัวอย่างกราฟของ sigmoid function}\label{fig:sigmoid}
  \source{[ที่มา : https://www.researchgate.net/figure/An-illustration-of-the-signal-processing-in-a-sigmoid-function_fig2_239269767]}
\end{figure}

\subsection{Rectified Linear Unit (ReLU)}
Rectified Linear Unit หรือ ReLU เป็น activate function ที่กำลังได้รับความนิยมเนื่องจากสามารถแก้ไขปัญหาในเรื่องของ 
anishing gradient problem ได้ เพราะกราฟของ ReLU นั้นถ้าค่า x เป็นบวกจะได้ค่าของอนุพันธ์เท่ากับ 1 เสมอทำให้ความชันไม่หาย 
ซึ่งทำให้ตัวโมเดลของเรานั้นปรับค่าน้ำหนักได้ไวยิ่งขึ้น แต่ก็มีข้อเสียเช่นกันคือผลลัพธ์จะออกมาอยู่ในช่วงตั้งแต่ 0 ถึง อินฟินิตี้ทำให้ไม่สามารถกำหนดขอบเขตได้ หรือผลลัพธ์สำหรับการที่ข้อมูลขาเข้าเป็นเลขติดลบจะเท่ากับ 
0 เสมอทำให้ไม่สามารถแปลงค่าผลลัพธ์ที่เท่ากับ 0 กลับมาเป็นข้อมูลขาเข้าได้ เป็นต้น 


\begin{figure}[!ht]\centering
  \setlength{\fboxrule}{0.2mm} % can define this in the preamble
  \setlength{\fboxsep}{1cm}
  \fbox{\includegraphics[width=5cm,height=3cm]{./relu.jpg}}
  \caption{ภาพตัวอย่างกราฟของ ReLU function}\label{fig:relu}
  \source{[ที่มา : https://www.researchgate.net/figure/ReLU-activation-function_fig7_333411007]}
\end{figure}



\subsection{การปรับขนาดรูปภาพ (Image rescale)}
การปรับขนาดรูปภาพ (Image rescale) นั้นเป็นส่วนหนึ่งของกระบวนการเตรียมข้อมูลเบื้องต้นสำหรับการทำโมเดล CNN 
เนื่องมาจากข้อมูลที่เราได้มาสำหรับการทำโมเดลนั้น อาจจะมีขนาดที่แตกต่างกันรวมถึงมีขนาดที่ใหญ่เกินไป ด้วยเหตุนั้นจะทำให้โมเดลใช้ระยะเวลาในการเรียนรู้นาน 
เราจึงกำหนดขนาดมาตรฐานและทำการปรับขนาดข้อมูลรูปภาพสำหรับการสร้างโมเดลก่อนที่จะนำไปใช้

\subsection{การแยกบริเวณรูปภาพ (Image segmentation)}
การแยกบริเวณรูปภาพ (Image segmentation) คือการแยกสิ่งที่เราสนใจออกมาจากพื้นหลังของรูปภาพเพื่อใช้ในการทำโมเดลต่อไป ซึ่งถูกนำไปใช้ประโยชน์ในหลายๆด้านด้วยกันได้แก่ 
การจับตัวหนังสือในภาพ การจับวัตถุแปลกปลอมในรูปภาพเป็นต้น โดยมีหลายรูปแบบด้วยกันยกตัวอย่างเช่น Region-Based Segmetation, Edge Detection Segmentation เป็นต้น
\begin{itemize}
  \item Region-Based Segmentation เป็นการแยกวัตถุออกจากภาพด้วยวิธีการใช้ค่า threshold เพื่อปรับภาพที่อยู่ในรูปแบบของ grayscale ให้กลายเป็น binary image โดยให้วัตถุเป็นหนึ่งสี และ พื้นหลังเป็นอีกสีหนึ่ง เพื่อที่เราจะได้รูปร่างของวัตถุขึ้นมา ซึ่งวิธีการเลือกค่า threshold ที่เหมาะสมมีมากมาย ยกตัยวอย่างเช่น Otsu’s thresholdig method

  \item Edge Detection Segmentation ที่ใช้ในการหาขอบของวัตถุซึ่งใช้หลักการความไม่ต่อเนื่องภายใน pixel ของรูปภาพ โดยจุดที่เกิดการเปลี่ยนแปลงนั้นจะถูกระบุเป็นจุดขอบ 
  \item Output layer เป็นชั้นสุดท้ายมีหน้าที่สำหรับรับข้อมูลจาก Hidden Layer เพื่อใช้ในการบอกว่าท้ายที่สุดแล้วตัวข้อมูลที่รับเข้ามานั้นถูกจำแนกอยู่ในประเภทใด
\end{itemize}
\begin{figure}[!ht]\centering
  \setlength{\fboxrule}{0.2mm} % can define this in the preamble
  \setlength{\fboxsep}{1cm}
  \fbox{\includegraphics[width=10cm]{./imagesegment.jpg}}
  \caption{ภาพตัวอย่างการทำ image segmentation บนภาพ}\label{fig:imagesegment}
  \source{[ที่มา : https://www.learnopencv.com/applications-of-foreground-background-separation-with-semantic-segmentation/]}
\end{figure}


\newpage
\subsection{การแปลงรูปภาพเป็นข้อความ (Optical character recognition)}
Optical character recognition หรือ OCR คือเทคโนโลยีที่ทำให้เราสามารถจับตัวอักษรที่อยู่ในภาพถ่ายยกตัวอย่างเช่น ภาพสแกนของเอกสาร หรือ สื่อสิ่งพิมพ์ต่างๆเป็นต้น มาแปลงให้อยู่ในรูปแบบของตัวอักษรดิจิตอล
 ที่สามารถแก้ไขได้ และง่ายต่อการจัดเก็บนำไปใช้ต่อ ซึ่งเราสามารถนำเทคโนโลยีไปประยุกต์ใช้ได้ในหลากหลายด้าน เช่น วิเคราะห์ทะเบียนรถยนต์ ด้านการทำระบบค้นหาข้อมูล หรือระบบจัดเก็บรายละเอียดสินค้า

\begin{figure}[!ht]\centering
  \setlength{\fboxrule}{0.2mm} % can define this in the preamble
  \setlength{\fboxsep}{1cm}
  \fbox{\includegraphics[width=10cm]{./ocr.jpg}}
  \caption{ภาพตัวอย่างขั้นตอนการแปลงเอกสารมาอยู่ในรูปแบบข้อมูลด้วยกระบวนการ OCR}\label{fig:ocr}
  \source{[ที่มา : https://medium.com/states-title/using-nlp-bert-to-improve-ocr-accuracy-385c98ae174c]}
\end{figure}

\subsection{Blob coloring}
ใช้ในการทำ OCR ของเราเป็นอัลกอริทึมที่ใช้ในการแบ่งขอบเขตของวัตถุ โดยจะทำการไล่ตั้งแต่ pixel บนสุดของภาพลงมาล่างสุดซึ่งแต่ละ
 pixel จะทำการจับว่า pixel รอบๆตัวนั้นเป็นสีดำหรือไม่ หากเป็นสีดำก็จะจับให้ pixel เหล่านั้นอยู่ใน label เดียวกัน ซึ่งวิธีการนี้จะทำให้เราสามารถแยกตัวอักษรแต่ละตัวออกจากกันได้ โดยแต่ละตัวก็จะมี label ของตัวมันเอง

\begin{figure}[!ht]\centering
  \setlength{\fboxrule}{0.2mm} % can define this in the preamble
  \setlength{\fboxsep}{1cm}
  \fbox{\includegraphics[width=8cm,height=5cm]{./blob.jpg}}
  \caption{ภาพตัวอย่างการทำงานของ Blob coloring}\label{fig:blob}
  \source{[ที่มา : http://what-when-how.com/introduction-to-video-and-image-processing/blob-analysis-introduction-to-video-and-image-processing-part-1/]}
\end{figure}



\section{Languages and technologies  ภาษาโปรแกรมและเทคโนโลยี}
เนื่องด้วยด้วยเป้าหมายของโครงการที่่ต้องการพัฒนาแอปพลิเคชันให้สามารถใช้งานได้ในหลายแพลตฟอร์ม 
โดยปัจจุบันระบบปฎิบัติการแอนดรอยด์และระบบปฎิบัติการไอโอเอสเป็นระบบปฎิบัติการที่ผู้คนใช้งานมากที่สุด 
โดยทั้งสองระบบปฎิบัติการครอบครองส่วนแบ่งทางตลาดมากกว่า 98\%  สำหรับโทรศัพท์มือถือและแท็ปเล็ต
\begin{figure}[!ht]\centering
  \setlength{\fboxrule}{0.2mm} % can define this in the preamble
  \setlength{\fboxsep}{1cm}
  \fbox{\includegraphics[width=10cm]{./market.jpg}}
  \caption{ส่วนแบ่งการตลาดระบบปฏิบัติการมือถือและแท็บเล็ตทั่วโลก}\label{fig:market}
  \source{[ที่มา : https://gs.statcounter.com/os-market-share/mobile-tablet/worldwide/#monthly-201907-202007]}
\end{figure}
จากสถิติระบบปฎิบัติการไอโอเอสและระบบปฎิบัติการแอนดรอยด์มีจำนวนผู้ใช้ปริมาณมาก ดังนั้นโครงการจึงพัฒนาแอปพลิเคชันให้สามารถใช้งานได้ทั้งสองระบบปฎิบัติการ 
ซึ่งรูปแบบในการพัฒนาแอปพลิเคชันให้สามารถใช้งานในหลายแพลตฟอร์มได้มีด้วยกันอยู่สองรูปแบบคือ Hybrid Application และ Web Application อย่างไรก็ตาม Hybrid Application 
สามารถทำงานได้ตามเป้าหมายของโครงการมากกว่าเพราะการเป็นรูปแบบแอปพลิเคชันทำให้สามารถใช้หน้าจอสัมผัสผ่านโทรศัพท์มือถือหรือแท็บเล็ตในการเขียนตัวอักษรได้ โดยมีเฟรมเวิร์คให้พัฒนามากมายเช่น React Native ,Ionic และ Flutter เป็นต้น


\subsection{React Native}
React Native คือ เฟรมเวิร์คที่พัฒนาด้วยภาษา JavaScript สำหรับการพัฒนาแอปพลิเคชันสำหรับระบบปฎิบัติการไอโอเอสและระบบปฎิบัติการแอนดรอยด์โดยใช้เทคโนโลยี 
Cross platform ซึ่งในการพัฒนาด้วย React Native มีข้อดีคือในการพัฒนาสามารถพัฒนาแค่ครั้งเดียวแต่สามารถใช้งานได้ทั้งระบบปฎิบัติการไอโอเอสและระบบปฎิบัติการแอนดรอยด์ 
ด้วยการจัดการของ JavaScript ให้สามารถสื่อสารกับฝั่ง Native ของระบบปฎิบัติการทั้งสอง จึงได้ผลลัพธ์ออกมาเป็น Native Application ทั้งระบบปฎิบัติการไอโอเอสและระบบปฎิบัติการแอนดรอยด์ 

\subsection{Keras}
Keras คือเฟรมเวิร์คที่พัฒนาด้วยภาษา Python ที่ถูกสร้างขึ้นมาเพื่อให้เราสามารถจัดการกับการทำ การเรียนรู้เชิงลึกของคอมพิวเตอร์ (deep learning) 
ได้อย่างง่าย ข้อดีของ Keras คือ ใช้งานง่าย และสามารถดัดแปลงตัว layer ของ neural network ได้ง่าย  โดยในโครงการนี้เราสามารถใช้ Keras 
ในการออกแบบตัว Layer ต่างๆ ของโมเดลได้รวมทั้งทำการสร้างโมเดลและทำนายด้วยภาษา Python ได้เลย

\subsection{OpenCV}
OpenCV เป็น library ที่มีจุดประสงค์เพื่อการแสดงผลด้วยคอมพิวเตอร์แบบเรียลไทม์  (Real - Time) รวมทั้งในส่วนของการทำ Image processing ที่รองรับการใช้งานบนหลายภาษาด้วยกัน
 โดยหนึ่งในนั้นคือ Python อีกทั้งยังสามารถใช้งานร่วมกับเฟรมเวิร์กการเรียนรู้เชิงลึกต่างๆ อาทิเช่น TensorFlow และ PyTorch  เป็นต้น 

\subsection{Django Rest Framework}
Django Rest Framework เป็น framework ที่พัฒนาขึ้นมาด้วยภาษา python ไว้ใช้สำหรับการสร้าง api ไว้คุยกับฐานข้อมูล  
เพื่อให้ website หรือ application สามารถเรียกใช้ตัว api เพื่อขอข้อมูล  ซึ่งสามารถพัฒนาได้ง่ายด้วยภาษา python
และที่เราเลือกเนื่องจากตัวโมเดลวินิจฉัยโรคของเราก็พัฒนาขึ้นจากภาษา python เช่นเดียวกันทำให้สามารถเรียกใช้งานได้ง่าย

\section{Related research / Competing solutions บทความที่เกี่ยวข้อง}

\subsection{Detecting Dyslexia Using Neural Networks}
\begin{figure}[!ht]\centering
  \setlength{\fboxrule}{0.2mm} % can define this in the preamble
  \setlength{\fboxsep}{1cm}
  \fbox{\includegraphics[width=6cm,height=5cm]{./dyslexia.jpg}}
\end{figure}

\subsubsection{การใช้ภาพลายมือในการวินิจฉัยโรค Dyslexia}
มีงานจำนวนมากที่วินิจฉัยโรค Dyslexia โดยการใช้ข้อมูลคะแนนการสอบ ประวัติของผู้ทดสอบ หรือ แบบสอบถาม และมีอีกจำนวนหนึ่งที่ใช้ข้อมูลเช่นภาพการทำงานของสมอง หรือ การสื่อสารผ่านทางดวงตาเป็นต้น 
แต่วรรณกรรมนี้ได้เลือกที่จะใช้ข้อมูลลายมือเนื่องจาก เป็นข้อมูลที่ง่ายต่อการเก็บรวบรวม


\subsubsection{การประมวลผลภาพ}
ในส่วนของการประมวลผลภาพนั้น วรรณกรรมนี้ได้ทำการนำภาพลายมือมาแบ่งเป็นบรรทัด 
หลังจากนั้นจึงได้นำแต่ละบรรทัดมาแบ่งเป็นอีก 50 ส่วน โดยวิธีการที่ใช้ในการแบ่งบรรทัด คือ 
 Arvanitopoulos & Susstrunk’s seam carving แล้วจึงนำภาพแต่ละบรรทัดไปแบ่งเป็น 50 
 ส่วนโดยใช้ขนาด 113*113 ซึ่งยังมีบางภาพที่ยากต่อการทำ แล้วจำเป็นต้องใช้การแก้ไขโดยผู้จัดทำก่อน แต่ไม่ได้มีความยากในการแก้ไขสูง

 \subsubsection{Optical character recognition}
 Optical character recognition หรือ OCR นั้นเป็นการจับตัวอักษรภายในภาพแล้วจึงนำมาแปลงเป็นค่า
  วิธีนี้สามารถอ่านได้ว่าในภาพนั้นมีตัวอักษรตัวใดอยู่บ้าง แต่จากการทดลองของวรรณกรรมนี้พบว่า 
  วิธีนี้ไม่เหมาะสมกับการนำมาอ่านภาพลายมือของเด็กที่เป็นโรค 
  เนื่องจากลายมือของเด็กนั้นมีความหลากหลายมาก ทำให้วิธีการตรวจจับด้วยระบบ OCR ไม่สามารถตรวจจับได้อย่างแม่นยำ
\subsubsection{การทำโมเดลวินิจฉัย}
เป็นส่วนที่ให้โมเดลนั้นได้ทำการระบุว่าข้อมูลที่ป้อนเข้ามาเป็นโรค dyslexia หรือไม่ 
โดยตัวโมเดลนั้นอยู่ในรูปแบบของ  โครงข่ายประสาทเทียมแบบสังวัตนาการ 
หรือ Convolutional Neural Network โดยประกอบด้วย convolutional layer จำนวน 5 layer 
max-pooling จำนวน 3 layer fully-connected จำนวน 2 layer และ dropout layer จำนวน 1 layer 
หลังจากนั้นจึงได้แบ่งข้อมูลแบบ 3:1:1 โดยเป็นข้อมูลในส่วนของการ train 60% test 20% และ validation 20% และได้ทดลองทำการเรียนรู้โมเดลด้วย batch size และจำนวนส่วนของแถวที่แบ่ง ด้วยหลายๆค่า โดยได้ค่าที่เหมาะสมคือ batch size = 4 และ แบ่ง 50 ส่วนต่อแถวของคำพูด

วรรณกรรมนี้ เป็นวรรณกรรมที่ดีและมีคล้ายกับว่ามีข้อเสนอแนะว่าไม่ควรใช้อะไรบ้าง 
รวมถึงช่วยเรื่องการคิดระบบการทำงานว่าควรมีขั้นตอนแบบใดจากวิธีแรกถึงวิธีสุดท้าย
 เห็นได้ว่ามีหลายวิธีอยากมากที่วินิจฉัยเรื่องของการเป็น LD แต่ว่าโปรเจ็คของเขาได้เลือกวิธีการวินิจฉัย
 ผ่านลายมือเนื่องจากสามารถเก็บรวบรวมได้ง่าย หลังจากนั้นนำภาพมาแบ่งเป็น 50 ส่วนตามขนาด 113*113 
 แต่ก็พบว่ายังมีบางภาพที่สามารถตัดแบ่งได้ยาก และใช้ OCR ในการระบุด้วย ผลออกมาคือมีความแม่นยำที่น้อย







%%%%%%%%%%%%%%%%%%%%%%%%%%%%%%%%%%%%%%%%%%%%%%%%%%%%%55
%%%%%%%%%%%%%%%%%%%%%%%%%%%%%%%%%%%%%%%%%%%%%%%%%%%%%
%%%%%%%%%%%%%%%%%%%%%%%%%%%%%%%%%%%%%%%%%%%%%%%%%%%%%
\chapter{วิธีการดำเนินงาน}

Explain the design (how you plan to implement your work) of your project. Adjust the section titles below to suit the types of your work. Detailed physical design like circuits and source codes should be placed in the appendix.

\section{Project Functionality}
\subsection{System Architecture}
\begin{figure}[!ht]\centering
  \setlength{\fboxrule}{0.2mm} % can define this in the preamble
  \setlength{\fboxsep}{1cm}
  \fbox{\includegraphics[width=6cm,height=5cm]{./system.jpg}}
  \caption{ภาพ System Architecture ของ LDSpot}\label{fig:system}
\end{figure}
\subsection{System requirements}
\begin{itemize}
  \item รองรับระบบปฏิบัติการแอนดรอยด์ตั้งแต่ 4.1 ขึ้นไป
  \item รองรับระบบปฏิบัติการไอโอเอสตั้งแต่ 10.0 ขึ้นไป
  \item รองรับระบบสัมผัสหน้าจอ
  \item สามารถดูผลการวินิจฉัยย้อนหลังได้
  \item อนุญาติให้เก็บผลการวินิจฉัยบนระบบได้
\end{itemize}
\subsection{(Use cases)}
\begin{figure}[!ht]\centering
  \setlength{\fboxrule}{0.2mm} % can define this in the preamble
  \setlength{\fboxsep}{1cm}
  \fbox{\includegraphics[width=6cm,height=5cm]{./usecase.jpg}}
  \caption{ภาพ Use Case Diagram}\label{fig:usecase}
\end{figure}
เมื่อผู้ใช้เข้าสู่แอปพลิเคชันของเราสิ่งแรกที่พบคือ หน้าหลัก (Main menu) เพื่อที่สามารถเชื่อมหรือใช้ฟังค์ชั่นอื่น ๆ โดยมี 4 ฟังค์ชั่น อย่างแรกเลย 
การวินิฉัย(Start Diagnosis) เมื่อผู้ใช้เลือกใช้ฟังค์ชั่นนี้ ทำให้เริ่มการวินิจฉัยโดยมีลักษณะคล้ายเกมส์ ให้เขียนตัวอักษร สระ และสะกดคำ จนเสร็จสมบูรณ์จากนั้น ก็วิเคราะห์ออกมาจากคำตอบที่เด็กได้ตอบระหว่างเกมส์ 
เพื่อให้ได้ผลลัพท์รวมถึง ส่งผลลัพธ์นั้นไปบอร์ดสถิติ (Dashboard) เพื่อที่แสดงข้อมูลให้ผู้ใช้คนอื่น ๆ เห็น นอกจากนี้ยังมีหน้าตั้งค่า(Setting) หน้าเกี่ยวกับเรา (About us) 

\section{โครงสร้างซอฟต์แวร์}
ในส่วนของการใช้งานระบบ LDSpot นั้นจะแบ่งเป็นสี่ส่วนหลักๆได้แก่ แอปพลิเคชันทำแบบทดสอบ การประมวลผลภาพ การแยกภาพ การวินิจฉัย โดยมีขั้นตอนของตัวระบบดังนี้
\begin{enumerate}
  \item ผู้ใช้จะต้องทำแบบทดสอบภายในแอปพลิเคชันโดยจะอยู่ในรูปแบบของเกมเขียน พยัญชนะ สระ และ สะกดคำ
  \item หลังจากนั้นภาพแบบทดสอบที่ผู้ใช้ได้ทำจะถูกส่งเข้าไปภายในระบบ LDSpot เพื่อทำการปรับปรุงคุณภาพของรูปภาพได้แก่การปรับขนาดของภาพให้เหมาะสม การลดสัญญาณรบกวนในภาพ และการปรับสีให้อยู่ในรูปแบบของขาวดำ
  \item เมื่อได้ภาพที่ผ่านการทำการปรับปรุงคุณภาพของภาพแล้ว ภาพจะถูกนำมาแบ่งเป็นช่องตามตัวอักษรโดยการสร้าง contour แล้วตีกรอบด้วย boundingbox ล้อมรอบแต่ละตัวอักษร หลังจากนั้นจึงตัดภาพตาม boundingbox ที่ได้สร้างไว้
  \item นำภาพแต่ละตัวอักษรเข้าไปวินิจฉัย เพื่อนำผลลัพธ์จากโมเดลมาแสดงผลบนแอปพลิเคชัน
\end{enumerate}

\subsection{แอปพลิเคชันทำแบบทดสอบ (Application)}
ในส่วนของแอปพลิเคชันทำแบบทดสอบนั้นเพื่อที่จะได้มาซึ่งภาพแบบทดสอบเราจึงออกแบบแอปพลิเคชันส์ในรูปแบบของเกมให้ผู้ใช้ทำ ซึ่งในส่วนนี้ผู้ใช้จะต้องทำแบบทดสอบการเขียนพยัญชนะ สระ และสะกดคำ โดยจะมีกรอบขึ้นมาให้ผู้ใช้เขียนตามเสียงพูด 
\begin{table}[!h]\centering
  \caption{แสดงข้อมูลขาเข้าและขาออกของ แอปพลิเคชัน}\label{tbl:application1}
  \begin{tabular}{c|c|l|rr} \hline
  Input & ผู้ใช้ทำแบบทดสอบภายในแอปพลิเคชัน \\ \hline
  Output & ภาพแบบทดสอบการเขียนพยัญชนะ สระ และ คำสะกด \\ \hline
  \end{tabular}
  \end{table}

\subsection{การประมวลผลภาพ (Image processing)}
ในส่วนนี้นั้นเราจะนำภาพแบบทดสอบที่ได้จากแอปพลิเคชันมาปรับปรุงคุณภาพของภาพเพื่อให้เหมาะสมแก่การนำไปวินิจฉัย 
โดยจะมีการปรับขนาดของภาพให้ตรงกับขนาดของภาพที่ระบบ LDSpot นั้นใช้ในการเรียนรู้ หลังจากนั้นจึงนำภาพไปทำการลดสัญญาณรบกวนด้วยวิธีการใช้ 
Gaussian blur และจึงปรับภาพให้อยู่ในสีขาวดำ เพื่อแยกตัวอักษรออกจากภาพพื้นหลัง
\begin{figure}[!ht]\centering
  \setlength{\fboxrule}{0.2mm} % can define this in the preamble
  \setlength{\fboxsep}{1cm}
  \fbox{\includegraphics[width=10cm]{./imageprocess.jpg}}
  \caption{แผนภาพแสดงข้อมูลขาเข้าและออกของการประมวลผลภาพ}\label{fig:system}
\end{figure}
\begin{table}[!h]\centering
  \caption{แสดงข้อมูลขาเข้าและขาออกของส่วนการประมวลผลภาพ}\label{tbl:application1}
  \begin{tabular}{c|c|l|rr} \hline
  Input & ภาพแบบทดสอบการเขียนพยัญชนะ สระ และ คำสะกด \\ \hline
  Output & ภาพแบบทดสอบการเขียนพยัญชนะ สระ และ คำสะกดที่ปรับปรุงคุณภาพสำหรับการทำโมเดลแล้ว \\ \hline
  \end{tabular}
  \end{table}
  

  \subsection{การแยกภาพ (Image segmentation)}
  ในส่วนนี้เราจะทำการสร้าง contour ขึ้นมาจากภาพที่ได้ทำการปรับสีขาวดำแล้ว โดยเราจะนำ contour นั้นไปสร้าง bounding box
   เพื่อครอบแต่ละตัวอักษรให้แยกออกจากกัน เนื่องจากเราต้องการภาพตัวอักษรที่อยู่เดี่ยวๆ ไปใช้ในการวินิจฉัยโรคบกพร่องทางการเรียนรู้ 
  \begin{figure}[!ht]\centering
    \setlength{\fboxrule}{0.2mm} % can define this in the preamble
    \setlength{\fboxsep}{1cm}
    \fbox{\includegraphics[width=10cm]{./imagesegment3.jpg}}
    \caption{แผนภาพแสดงข้อมูลขาเข้าและออกของการแยกภาพ}\label{fig:system}
   \end{figure}
  \begin{table}[!h]\centering
    \caption{แสดงข้อมูลขาเข้าและขาออกของส่วนการแยกภาพ}\label{tbl:application1}
    \begin{tabular}{c|c|l|rr} \hline
    Input & ภาพแบบทดสอบการเขียนพยัญชนะ สระ และ คำสะกดที่ปรับปรุงคุณภาพสำหรับการทำโมเดลแล้ว  \\ \hline
    Output & ภาพที่ถูกตัดให้เหลือเพียงภาพละ 1 พยัญชนะ สระ หรือ คำสะกด เพื่อนำไปเข้าระบบ LDSpot เพื่อวินิจฉัย\\ \hline
    \end{tabular}
    \end{table}

  \newpage
  \subsection{การวินิจฉัย (Learning disorder prediction)}
  เมื่อเราได้ภาพตัวอักษรเดี่ยวๆจากส่วนการแยกภาพแล้ว เราจะนำภาพตัวอักษรเดี่ยวๆนั้นไปโยนเข้าโมเดลที่เราได้ทำการสร้างไว้ 
  เพื่อให้โมเดลวินิจฉัยโรคบกพร่องทางการเรียนรู้ แล้วนำผลลัพธ์ส่งกลับไปในฐานข้อมูลเพื่อให้แอปพลิเคชันสามารถเรียกข้อมูลไปแสดงได้ต่อไป 
  \begin{figure}[!ht]\centering
    \setlength{\fboxrule}{0.2mm} % can define this in the preamble
    \setlength{\fboxsep}{1cm}
    \fbox{\includegraphics[width=10cm]{./imagepredict3.jpg}}
    \caption{แผนภาพแสดงข้อมูลขาเข้าและออกของการวินิจฉัย}\label{fig:system}
   \end{figure}
  \begin{table}[!h]\centering
    \caption{แสดงข้อมูลขาเข้าและขาออกของส่วนการวินิจฉัย}\label{tbl:application1}
    \begin{tabular}{c|c|l|rr} \hline
    Input & ภาพที่ถูกตัดให้เหลือเพียง ภาพละ 1 พยัญชนะ สระ หรือ คำสะกด เพื่อนำไปเข้าระบบ LDSpot เพื่อวินิจฉัย  \\ \hline
    Output & ผลลัพธ์จากการวินิจฉัยในรูปแบบของคะแนนและ โอกาสการเป็นโรคบกพร่องทางการเรียนรู้ \\ \hline
    \end{tabular}
    \end{table}

\section{Conceptual  Design}
\begin{figure}[!ht]\centering
  \setlength{\fboxrule}{0.2mm} % can define this in the preamble
  \setlength{\fboxsep}{1cm}
  \fbox{\includegraphics[width=10cm]{./conceptual.jpg}}
  \caption{ภาพการสื่อสารระหว่างทางฝั่ง Frontend และ Backend}\label{fig:conceptual}
 \end{figure}
 การทำงานของตัวระบบ LDSpot นั้นจะมีอยู่สองส่วนด้วยกันได้แก่ front-end ที่ทำหน้าที่เป็นหน้าแอปพลิเคชันไว้สื่อสารกับผู้ใช้งาน และส่วนของ back-end ที่รับข้อมูลมาเพื่อประมวลผลแล้วหลังจากนั้นจึงนำข้อมูลไปเก็บใส่ฐานข้อมูลไว้ใช้งานต่อไป 
 \begin{itemize}
   \item ในส่วนของ front-end นั้นจะประกอบไปด้วย แอปพลิเคชันส์ในรูปแบบของเกม โดยที่ผู้ใช้จะสามารถเข้าสู่ระบบผ่านทางรหัสที่ได้ทำการสมัครสมาชิกไว้
    โดยตัวรหัสจะถูกส่งไปเพื่อตรวจสอบความถูกต้องว่ามีรหัสนี้อยู่จริงในระบบหรือไม่กับฐานข้อมูลที่อยู่ภายในส่วนของ back-end หากตรวจสอบแล้วถูกต้องจึงจะสามารถเข้าสู่ระบบได้ 
   \item ผู้ใช้สามารถกดเข้ารับแบบทดสอบได้ โดยเมื่อเข้ารับแล้ว จะต้องทำตามข้ันตอนต่อไป ซึ่งผู้ใช้จะได้เขียนตัวอักษร สระ และคำสะกด ตามเสียงไปเรื่อยๆ เมื่อเสร็จสิ้นแล้วภาพแบบทดสอบจะถูกส่งไปทางฝั่ง back-end 
   ในส่วนของระบบ LDSpot เพื่อทำการวินิจฉัยหลังจากนั้นผลลัพธ์จะถูกส่งเก็บเข้าไปในฐานข้อมูล เพื่อให้ทางฝั่ง front-end สามารถดึงข้อมูลไปแสดงผลบนแอปพลิเคชันได้
   \item ผู้ใช้สามารถดูผลลัพธ์ย้อนหลังได้โดยกดดูผลลัพธ์ภายในแอปพลิเคชันหลังจากนั้น แอปพลิเคชันจะทำการติดต่อกับฐานข้อมูลเพื่อดึงผลลัพธ์ที่เคยได้มีการวินิจฉัยไว้ของผู้ใช้งานคนนั้นมาแสดงผล
   \item ผู้ใช้สามารถดูบอร์ดสถิติได้ โดยบอร์ดสถิตินั้นจะดึงข้อมูลสรุปจากฐานข้อมูลมาว่า มีผู้ใช้งานแอปพลิเคชันนี้แล้วกี่คน มีการทำนายว่าเป็นโรคบกพร่องทางการเรียนรู้กี่คน ไม่เป็นกี่คนเป็นต้น เพื่อใช้เป็นข้อมูลในเชิงสถิติต่อไป
 \end{itemize}

 
\section{Database Design}
\begin{figure}[!ht]\centering
    \setlength{\fboxrule}{0.2mm} % can define this in the preamble
    \setlength{\fboxsep}{1cm}
    \fbox{\includegraphics[width=10cm]{./database.jpg}}
    \caption{ภาพ Database ER diagram}\label{fig:database}
   \end{figure}
   
\newpage
\section{Activity Diagram Design}
   \begin{itemize}
    \item ทำแบบทดสอบ
    \begin{figure}[!ht]\centering
      \setlength{\fboxrule}{0.2mm} % can define this in the preamble
      \setlength{\fboxsep}{1cm}
      \fbox{\includegraphics[width=10cm]{./activitytest.jpg}}
      \caption{ภาพ Activity diagram การทำแบบทดสอบ}\label{fig:activity1}
     \end{figure}
    \newpage
    \item ดูผลลัพธ์การทดสอบ
    \begin{figure}[!ht]\centering
      \setlength{\fboxrule}{0.2mm} % can define this in the preamble
      \setlength{\fboxsep}{1cm}
      \fbox{\includegraphics[width=10cm]{./activityresult.jpg}}
      \caption{ภาพ Activity diagram การดูผลลัพธ์การทดสอบ}\label{fig:activity2}
     \end{figure}
     \newpage
    \item ดูสถิติรวมของแอปพลิเคชัน
    \begin{figure}[!ht]\centering
      \setlength{\fboxrule}{0.2mm} % can define this in the preamble
      \setlength{\fboxsep}{1cm}
      \fbox{\includegraphics[width=10cm]{./activitystat.jpg}}
      \caption{ภาพ Activity diagram การดูข้อมูลสถิติในแอปพลิเคชัน}\label{fig:activity3}
     \end{figure}
  \end{itemize}
 











\newpage
\section{User Interface Design}
\begin{itemize}
  \item Login
  \begin{figure}[!ht]\centering
    \setlength{\fboxrule}{0.2mm} % can define this in the preamble
    \setlength{\fboxsep}{1cm}
    \fbox{\includegraphics[width=10cm]{./login.jpg}}
    \caption{ภาพการออกแบบหน้า Login}\label{fig:system}
  \end{figure}
  \item Home
  \begin{figure}[!ht]\centering
    \setlength{\fboxrule}{0.2mm} % can define this in the preamble
    \setlength{\fboxsep}{1cm}
    \fbox{\includegraphics[width=10cm]{./home.jpg}}
    \caption{ภาพการออกแบบหน้า Home}\label{fig:system}
  \end{figure}
  \newpage
  \item TestStage1
  \begin{figure}[!ht]\centering
    \setlength{\fboxrule}{0.2mm} % can define this in the preamble
    \setlength{\fboxsep}{1cm}
    \fbox{\includegraphics[width=10cm]{./stage1.jpg}}
    \caption{ภาพการออกแบบหน้าการทำแบบทดสอบด่านแรก}\label{fig:system}
  \end{figure}
  \item TestStage2
  \begin{figure}[!ht]\centering
    \setlength{\fboxrule}{0.2mm} % can define this in the preamble
    \setlength{\fboxsep}{1cm}
    \fbox{\includegraphics[width=10cm]{./stage2.jpg}}
    \caption{ภาพการออกแบบหน้าการทำแบบทดสอบด่านสอง}\label{fig:system}
  \end{figure}
  \newpage
  \item TestStage3
  \begin{figure}[!ht]\centering
    \setlength{\fboxrule}{0.2mm} % can define this in the preamble
    \setlength{\fboxsep}{1cm}
    \fbox{\includegraphics[width=10cm]{./stage3.jpg}}
    \caption{ภาพการออกแบบหน้าการทำแบบทดสอบด่านสาม}\label{fig:system}
  \end{figure}
  \item Skip
  \begin{figure}[!ht]\centering
    \setlength{\fboxrule}{0.2mm} % can define this in the preamble
    \setlength{\fboxsep}{1cm}
    \fbox{\includegraphics[width=10cm]{./stage1inputSkip.jpg}}
    \caption{ภาพการออกแบบหน้าการกดข้ามการเขียนตัวอักษร สระ และคำสะกด}\label{fig:system}
  \end{figure}
  \newpage
  \item Writing Scene
  \begin{figure}[!ht]\centering
    \setlength{\fboxrule}{0.2mm} % can define this in the preamble
    \setlength{\fboxsep}{1cm}
    \fbox{\includegraphics[width=10cm]{./stage1input.jpg}}
    \caption{ภาพการออกแบบหน้าเขียนตัวอักษร สระ และคำสะกด}\label{fig:system}
  \end{figure}
  \begin{figure}[!ht]\centering
    \setlength{\fboxrule}{0.2mm} % can define this in the preamble
    \setlength{\fboxsep}{1cm}
    \fbox{\includegraphics[width=10cm]{./stage1input2.jpg}}
    \caption{ภาพการออกแบบหน้าเขียนตัวอักษร สระ และคำสะกด}\label{fig:system}
  \end{figure}
  \newpage
  \item EndScene
  \begin{figure}[!ht]\centering
    \setlength{\fboxrule}{0.2mm} % can define this in the preamble
    \setlength{\fboxsep}{1cm}
    \fbox{\includegraphics[width=10cm]{./endGame.jpg}}
    \caption{ภาพการออกแบบหน้าจบการทดสอบ}\label{fig:system}
  \end{figure}s
  \item Result
    \begin{figure}[!ht]\centering
      \setlength{\fboxrule}{0.2mm} % can define this in the preamble
      \setlength{\fboxsep}{1cm}
      \fbox{\includegraphics[width=10cm]{./result.jpg}}
      \caption{ภาพออกแบบหน้าดูผลลัพธ์การทดสอบ}\label{fig:system}
    \end{figure}
  \newpage
  \item Stat
    \begin{figure}[!ht]\centering
      \setlength{\fboxrule}{0.2mm} % can define this in the preamble
      \setlength{\fboxsep}{1cm}
      \fbox{\includegraphics[width=10cm]{./stat.jpg}}
      \caption{ภาพการออกแบบหน้าดูสถิติภายในแอปพลิเคชัน}\label{fig:system}
        
  \end{figure}
\end{itemize}
\newpage
\section{การเก็บข้อมูลภาพลายมือเด็ก}
การเก็บรวมรวมข้อมูลของภาพลายมือเด็ก เราได้ทำการรวมรวมรูปภาพการทำแบบทดสอบโดยเขียนพยัญชนะ สระ และคำสะกด 
จากนักเรียนระดับชั้นประถมศึกษาตั้งแต่ประถมศึกษาปีที่หนึ่งถึงประถมศึกษาปีที่สาม โดยคาดว่าจะมีเด็กเข้าร่วมทำแบบทดสอบประมาณ 1000 
คนโดยประมาณ ซึ่งภาพลายมือเด็กที่เขียนถูกต้องจะถูกนำมาใช้ในการเรียนรู้ของตัวโมเดลของเรา โดยตัวอย่างแบบทดสอบมีดังนี้
\begin{figure}[!ht]\centering
  \setlength{\fboxrule}{0.2mm} % can define this in the preamble
  \setlength{\fboxsep}{1cm}
  \fbox{\includegraphics[width=10cm]{./พยัญชนะ.png}}
  \caption{ภาพแบบทดสอบที่ใช้เก็บลายมือตัวอักษรเด็ก}\label{fig:system}
    
\end{figure}
\begin{figure}[!ht]\centering
  \setlength{\fboxrule}{0.2mm} % can define this in the preamble
  \setlength{\fboxsep}{1cm}
  \fbox{\includegraphics[width=10cm]{./สระ.png}}
  \caption{ภาพแบบทดสอบที่ใช้เก็บลายมือสระเด็ก}\label{fig:system}
    
\end{figure}
\begin{figure}[!ht]\centering
  \setlength{\fboxrule}{0.2mm} % can define this in the preamble
  \setlength{\fboxsep}{1cm}
  \fbox{\includegraphics[width=10cm]{./คำสะกด.png}}
  \caption{ภาพแบบทดสอบที่ใช้เก็บลายมือคำสะกดเด็ก}\label{fig:system}
    
\end{figure}



%%%%%%%%%%%%%%%%%%%%%%%%%%%%%%%%%%%%%%%%%%%%%%%%%%%%%%%%%%%%%%%
%%%%%%%%%%%%%%%%%%%% Bibliography %%%%%%%%%%%%%%%%%%%%%%%%%%%%%
%%%%%%%%%%%%%%%%%%%%%%%%%%%%%%%%%%%%%%%%%%%%%%%%%%%%%%%%%%%%%%%


%%%% Comment this in your report to show only references you have
%%%% cited. Otherwise, all the references below will be shown.
\nocite{*}
%% Use the kmutt.bst for bibtex bibliography style 
%% You must have cpe.bib and string.bib in your current directory.
%% You may go to file .bbl to manually edit the bib items.
\bibliographystyle{kmutt}
\bibliography{string,cpe}



\end{document}
